\section{Einleitung}
Ich beschäftige mich im Folgenden mit den jüngsten Ereignissen des Sharehosters MegaUpload. Ein
Sharehoster ist ein Internetdienst, der es erlaubt, Dateien begrenzter Größe ohne eine vorherige
Anmeldung hochzuladen. Diese Dateien können dann über eine URL wieder heruntergeladen werden. Solche
Dienste können beispielsweise für den Austausch größerer Dateien genutzt werden, die nicht mehr als
Email Anhang vom Email-Anbieter akzeptiert werden. Oder auch um eine kostenlose Sicherheitskopie der
eigenen, hoffentlich verschlüsselten Daten zu erstellen.
Doch die meisten Sharehoster werden ohne Frage hauptsächlich für Urheberrechtsverletzungen genutzt.
Eine der weltweit bekanntesten Sharehoster ist beziehungsweise war MegaUpload.

\section{Überblick}
Am Abend (deutscher Zeit) des \printdate{19.01.2012} wurden Kim Schmitz
und drei weitere Beschuldigte vom FBI verhaftet
und es wurden 20 Hausdurchsuchungen durchgeführt (hauptsächlich in den USA). Die
Inhaftierung von drei weiteren Personen steht noch aus. Die Anklage lautet auf Urheberverletzungen
in großem Stil sowie Geldwäsche. Über die Verhaftung von Kim Schmitz ist noch zu sagen, dass er
gerade seinen 38. Geburtstag in seiner Villa feierte und zu dieser Party waren Unteranderem auch
drei weitere, von US-Behörden gesuchte Personen anwesend. Der Zeitpunkt für die länderübergreifende
Verhaftung der MegaUpload Betreiber war recht geschickt gewählt. Dies war kein Zufall, den
Ermittlern war vorher bekannt, dass die drei namentlich nicht genannten extra für die
Geburtstagsparty anreisen würden. Zudem lässt sich vermuten, dass das FBI ihre Ermittlungen auch auf
einen optimalen PR-Effekt ausgelegt haben. So suchten sie sich, dass viel umworbene und in der
Kritik stehende Unternehmen MegaUpload aus.
Am selben Tag wurden auch die, in verschiedenen Rechenzentren der Welt untergebrachten
MegaUpload-Server, vom Netz genommen.

Nach drei Tagen beantragte Kim Schmitz alias \enquote{Kim Dotcom}\footnote{auch bekannt als
\enquote{Kimperator} oder \enquote{Kimble}} Freilassung auf Kaution, die ihm aber nicht
gewährt wurde.
Laut Staatsanwaltschaft liegt bei Kim Schmitz Fluchtgefahr vor. Dies ergab sich durch ein
Skype-Chat-Protokoll von 2007, dass vom FBI angefertigt wurde. Die Befürchtung vom neuseeländischen
Untersuchungsrichter ist nun, dass Kim Schmitz noch über Privateigentum verfügt, dass nicht
eingefroren wurde. Dieses Geld könnte Kim Schmitz zur Fluch nach Deutschland benutzen, um eine
Anhörung in den USA zu verhindern. Da Deutschland seine Staatsbürger nicht an die USA ausliefert.
Die Verteidigung von Kim Schmitz argumentierte dagegen, da der Angeklagte eine Frau und drei Kinder
hat.

Die Reaktionen auf die Razzia gegen das MegaUpload Imperium waren nicht zu übersehen. So gab es
DDoS-Attacken\footnote{Eine distributed denial-of-service Attacke besteht aus massenhaften Anfragen
auf ein Informationssystem mit dem Ziel der Überlastung des Systems. Diese Anfragen gehen von vielen
einzelnen Computern aus. Die anfragenden Computer sind meist (unfreiwilliges) Mitglied eines
\href{http://de.wikipedia.org/wiki/Botnet}{Botnetzes}.}
die sich gegen die Webseite des US-Justizministerium richteten. Verantwortlich für die Angriffe
sind vermutlich Anonymous-Aktivisten. Zudem
sackte nach einer Statistik von Arbor Networks zufolge der weltweite Internet-Traffic unmittelbar
nach dem Abschalten von MegaUpload deutlich ab\cite{arbornetworks:MegaUpload_shutdown_effect}.
Es laufen Initiativen um die Dateien auf den Servern
von MegaUpload zu erhalten beziehungsweise es den Nutzern zu ermöglichen ihre eigenen (legalen)
Inhalte wieder herunterzuladen. Weiterhin brach eine große Panik bei vielen Konkurrenten von
MegaUpload aus. So stellten einige ihre Dienste (zeitweise) komplett ein oder löschten massenhaft
Dateien. Die Panik
beschränkte sich nicht nur auf die Betreiber, insbesondere die aktiven, Premium Nutzer dürften etwas
nachdenklich geworden sein, falls sie sich nicht vorher Gedanken gemacht haben. Bei besonders
schweren Fällen ist eine Ermittlung durchaus noch möglich.
Aber neben all der Panik ist auch klar dass die Piraterie weiter gehen wird.

Die Abschaltung von MegaUpload wird nun dazu benutzt Gesetzesvorschlägen wie SOPA (Stop Online
Piracy Act) oder PIPA (Protect IP Act) umzusetzen die das Ziel haben gegen Piraterie vorzugehen.
Die Ermittlung gegen MegaUpload zeigt aber, dass die US-Behörden gar nicht so Handlungsunfähig
sind, wie sie zur Begründung der beiden Gesetze beklagten.

\begin{description}
	\item[Aktualisierung vom \printdate{22.02.2012}:] Kim Schmitz ist in Neuseeland auf Kaution
		freigekommen. Die befürchtete Fluchtgefahr wurde nicht bestätigt. Er muss aber strenge
		Auflagen befolgen, unter anderem wird ihm der Zugriff auf das Internet verwehrt
		\cite{golem:Kim_Frei,sueddeutsche:Kim_Frei}.
\end{description}

\section{Problem MegaUpload?}
Als Sharehoster ist MegaUpload erst einmal nur ein Diensteanbieter und ist somit neutral. Ähnlich
wie die Post, die bei Briefbomben auch nicht haftet. Allerdings wird schon länger versucht, diesen
Sachverhalt umzudrehen. Da es sehr schwer ist etwas gegen die Nutzer von Sharehostern zu
unternehmen, da anders als
bei Tauschbörsen, die IP-Adresse eines Nutzers nie
direkt bei einem anderen Nutzern auftaucht, sondern nur beim Sharehoster und diese dürften ein
gewisses
Interesse daran haben keine Logdateien zu erstellen, um mögliche illegale Aktivitäten nicht zu
protokollieren. Soviel zur Theorie, kommen wir zu MegaUpload.

MegaUpload hat sich durch mehrere Punkte angreifbar gemacht, die nun gegen das Unternehmen
hervorgebracht werden.
Diese sind in der 72-seitigen Anklageschrift \cite{Anklageschrift} der USA aufgeführt.
So steht in
der Kritik das MegaUpload von Verstößen gegen das Urheberrecht gewusst hat und diese sogar förderte.
MegaUpload gab Prämien für Nutzer aus, deren Dateien besonders beliebt waren und diese Dateien sind
in der Regel die neusten Kinofilme oder ähnliche Inhalte -- also auf jeden Fall urheberrechtlich
geschütztes Material. Über diese Urheberrechtsverstöße hat auch MegaUpload indirekt, über Werbung,
mitverdient. Weiterhin soll MegaUpload, Dateien, die als illegal gemeldet wurden, nicht vollständig
gelöscht haben. Damit ist Folgendes gemeint. Wenn Nutzer A eine Datei hochgeladen hat und die
gleiche Datei später auch von Nutzer B hochgeladen wird, dann wird auf den Servern die Datei nur
einmal gespeichert\footnote{Es ist also eine Deduplizierung auf Dateiebene mithilfe von
MD5-Hashes. Der Hauptgrund hierfür ist die Speicherplatz-Ersparnis.}.
MegaUpload erstellt aber dennoch zwei verschiedene URLs (A und B) für diese Datei. Wird die
Datei nun gemeldet, wurde nur die eine URL (B) blockiert, aber über die URL (A) ist die Datei
weiterhin abrufbar. Dies ist allerdings kein technisches Problem, wie
MegaUpload in Fällen von Kinderpornografie bewiesen hat, da in diesen Fällen alle Links deaktiviert
wurden.
Weiterhin sind einige Emails der Verdächtigen in der Anklageschrift veröffentlicht, die
Urheberechtsverletzungen der Verfasser nahe legen.

Also für die Betreiber und die Anwälte eine nicht ganz leichte Situation.

\bigskip
Abschließend kann man sich noch fragen, ob nicht die Filmindustrie ihre 500 Millionen US-Dollar
Schaden \cite{Anklageschrift} über einen Vertriebsweg wie MegaUpload auch selbst einnehmen kann.
Und damit dem Kunden die Werke für einen angemessenen Preis direkt ins Wohnzimmer liefern könnte.
